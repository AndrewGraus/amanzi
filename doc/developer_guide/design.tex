% -------------------------------------------------------------------------
%  This is a good place to outline key high-level design principled
% -------------------------------------------------------------------------

\section{Design of Amanzi}

In short, the guiding  principles described below are related to code readability,
modularity, and extensibility.
We facilitate community code development and code review, we follow closely, but not exactly, 
the Google C++ coding style, see 
\begin{center}
{\tt https://google.github.io/styleguide/cppguide.html}
\end{center}

for more detail.
In this section, we describe high-level principles and elaborate some of them 
in the subsequent section.
We use different fonts to distinguish between a {\tt Class} name, its {\it Methods()}, 
and its {\it variables}. 
Global {\rm CONSTANTS} are capitalized.

\underline{Disclaimer}. Amanzi's initial code implementation of new models 
and algorithms does not always comply with the formulated design principles, but 
it is getting there with each code re-factory.


%%%%%%%%%%%%%%%%%%%%%%%%%%%%%%%%%%%%%%%%%%%%%%%%%%%%%%%%%%%%%%%%%%%%%%
\subsection{State}
State is a simple data manager. 
It allows process kernels (PK) to require, read, and write various variables (such as physical fields).
It guarantees data protection by providing both const and non-const data pointers for variables.
It provides some initialization capability -- this is where all independent variables can be 
initialized -- since independent variables are typically owned by the state, not by a process kernel.
A few initialization tools are supported: a space-time function, initialization from an Exodus file 
and initialization from an HDF5 file.



%%%%%%%%%%%%%%%%%%%%%%%%%%%%%%%%%%%%%%%%%%%%%%%%%%%%%%%%%%%%%%%%%%%%%%
\subsection{PK and MPC PK}
PK stands for the Process Kernel.
MPC stands for the Multi-Process Coupler.
Each PK and MPC PK does little actual numerical work.
Instead, PK administrates discretization schemes, time integrators, and solvers. 
Each PK may represent a single equation (e.g. the Poisson equation for the Darcy flow) 
or system of strongly connected equations (e.g. the Navier-Stokes flow).

An MPC PK couples multiple physical processes which have their respected PKs.
One example is the Darcy flow and dispersive transport of chemical components.
An MPC PK may often be fully automated with no knowledge of the underlying PKs.
Since an MPC PK has the same interface as a PK, it is also a process kernel which
allows us to build a hierarchy of physical models with various degree of coupling
ranging from a weak coupling to an iterative coupling to a strong coupling.


Much of the work in a PK is delegated to field evaluators, which implement various 
physical and mathematical models, such as the equations of state, or boundary conditions, 
or mesh deformation. 
For these reasons, it is appropriate to call them variable evaluators.
The available variable evaluators are classified as follows:

\begin{enumerate}
\item Independent variable evaluators are the user-provided functions of spatial and temporal coordinates
      and has no dependencies.
      They could be used to compute (analytic or tabulated) boundary conditions, source terms, and initial conditions. 
\item Primary variable evaluators are related to the fields solved for within a PK.
      Examples are pressure and temperature fields.
      Typically these evaluators are used internally to track change in fields state and inform the 
      dependency tree about this.
\item Secondary variable evaluators are derived either from primary variable evaluators or other secondary variables. 
      There are two types of the secondary variable evaluators used to evaluate either a single or multiple variables.
      A model for a secondary variable can be anything from a constitutive relation to a discrete operator
      (apply a divergence operator to a velocity given a mesh and discretization) 
      to a summation operator (add the divergence of Darcy fluxes to a source term to determine the mass balance).
      Quite often, the secondary field/variable evaluators are created by high-level PKs during the setup phase 
      and inserted automatically in the list of evaluators. 
\end{enumerate}

The evaluator is much like a functor or function; it stores no actual data, only meta-data and 
a few parameters or constants.
It accesses data using a data manager, which controls access for both read-only and read/write modes. 

All evaluators are stored in a dependency graph, which is a directed, acyclic graph (DAG) 
describing the functional relationship of each variable in the state. 
End nodes in the dependency graph are either independent variables or primary variables. 
All other nodes in the graph are secondary variables.

The combination of a data manager and a dependency graph enables dynamic definition of each variable's model 
and data, and splits complex equations into manageable chunks. 
It also allows lazy evaluation, where nodes in the graph are updated (re-calculated) only if their dependencies
have changed, resulting in a managed, automated evaluation process with fewer bugs and inefficiencies.
For more details, we refer to \cite{coon2016managing}.
A developer may easily modify behavior of evaluators by overriding virtual member functions. 
The example below shows implementation of one function in an abstract product evaluator of type 
$\Pi_{i=1}^N f_i^{p_i}$ where $p_i$ is either 1 or -1.

\begin{lstlisting}[language=C++]
void ProductEvaluator::EvaluateField_(
    const Teuchos::Ptr<State>& S,
    const Teuchos::Ptr<CompositeVector>& result)
{
  auto& result_c = *result->ViewComponent("cell");
  int ncells = result_c.MyLength();

  int n(0);
  result_c.PutScalar(1.0);
  for (auto it = dependencies_.begin(); it != dependencies_.end(); ++it) {
    const auto& field = *S->GetFieldData(*it)->ViewComponent("cell");
    if (powers_[n] == 1)
      for (int c = 0; c != ncells; ++c) result_c[0][c] *= field[0][c];
    else if (powers_[n] == -1)
      for (int c = 0; c != ncells; ++c) result_c[0][c] /= field[0][c];
    n++;
  }
}
\end{lstlisting}



%%%%%%%%%%%%%%%%%%%%%%%%%%%%%%%%%%%%%%%%%%%%%%%%%%%%%%%%%%%%%%%%%%%%%%
\subsection{CompositeVector}
Class {\tt CompositeVector} is an implementation of an improved
{\tt Epetra\_MultiVector} (from Trilinos suite of packages) which spans multiple components and knows how to
communicate itself.
A composite vector is a collection of vectors defined on a common mesh and
communicator. 
Each vector, or component, has a name (typically, a mesh entity)
and a number of degrees of freedom.  
This meta data is stored in class {\tt CompositeVectorSpace}.
For instance, the field {\it total\_component\_concentration} is the cell-centered field with 
as many degrees of freedom as there exist chemical components.

Ghost cell updates are managed by the class {\tt CompositeVector}. 
Design of the parallel communication strategy is driven by two observations:
\begin{itemize}
\item The need for updated ghost cell information is typically known by the
      user just prior to being used, not just after the master values are
      updated.
\item Occasionally multiple functions need ghost values, but no changes to
      owned data have been made between these functions.  However, it is not
      always possible for the second call to know, for certain, that the first
      call did the communication.  Versatility means many code paths may be
      followed.
\end{itemize}


\subsubsection{Parallel communications}
To avoid unnecessary parallel communication the following algorithms were implemented
but are not active now.
This may change in the future.

Each time the vector values are changed, an internal flag is marked to
record that the ghost values are stale.
Each time ghost cells are needed, that flag is checked and communication
is done, if needed.
Keeping this flag correct is therefore critical. 
To do this, access to vectors must follow the rigid pattern.
The following modifications tag the flag:

\begin{enumerate}
\item Any of the usual {\it PutScalar()}, {\it Apply()}, etc methods.
\item Non-const calls of {\it ViewComponent()}.
\item Call of {\it GatherMasterToGhosted()} and {\it ChangedValues()}.
\item {\it Scatter()} called in a non-{\rm INSERT} mode.
\end{enumerate}

There exist known ways to break this paradigm. 
One is to store a non-const pointer to the underlying {\tt Epetra\_MultiVector}.
The fix is simple as this: \underline{never} store a pointer to the underlying data, 
just keep pointers to the composite vector itself.

The other one is when one grabs a non-const pointer, calls {\it Scatter()}, then 
changes the values of the local data.  
This is the nasty one, because it is both subtle and reasonable usage.
When you access a non-const pointer, the data is flagged as changed.
Then you call {\it Scatter()} and the data is flagged as unchanged.
Then you change the data from your old non-const pointer, so that the data is changed, but not flagged.
The first fix is to always call {\it ViewComponent()} after {\it Scatter()} and before changing values.
Another way to protect yourself is to put non-const references in their own scope.
For instance, the following practice is encourage:
\begin{lstlisting}[language=C++]
CompositeVector my_cv;
{ // unnamed scope for my_vec
  Epetra_MultiVector& my_vec = *my_cv.ViewComponent("cell", false);
  my_vec[0][0] = 12;
} // close scope of my_vec

my_cv.ScatterMasterToGhosted()

// Reference to my_vec is now gone, so we cannot use it and screw things up!

{ // unnamed scope for my_vec
  // This is now safe!
  Epetra_MultiVector& my_vec = *my_cv.ViewComponent("cell", true);
  my_vec[0][0] = my_vec[0][ghost_index] + ...
} // close scope of my_vec
\end{lstlisting}

The final way to break the parallel machinery is to use {\it const\_cast()} and 
then change the values.
Const-correctness is your friend. Keep your PKs const-correct, and you will never have this problem.

Note that the non-{\rm INSERT} modes of scatter are never skipped because of the flag state, 
and the flag is always tagged as changed.  
This is because subsequent calls with different modes would break the code.


%%%%%%%%%%%%%%%%%%%%%%%%%%%%%%%%%%%%%%%%%%%%%%%%%%%%%%%%%%%%%%%%%%%%%%
\subsection{TreeVector}
The class {\tt TreeVector} implements a nested, hierarchical data structure 
that mimics that for PK hierarchies.
It is an extendable collection of composite vectors.
This vector allows each physical PK to use composite vector to store 
their solution, and allows MPCs to push back vectors in a tree format.

This class provides the standard vector interface (extended ring algebra) and 
may be used with time integrators and nonlinear solvers.


%%%%%%%%%%%%%%%%%%%%%%%%%%%%%%%%%%%%%%%%%%%%%%%%%%%%%%%%%%%%%%%%%%%%%%
\subsection{Linear operators}
The idea behind the design of Amanzi operators is to separate three 
functionalities that are frequently placed in a single class in other
C++ packages.

\begin{enumerate}
\item Containers of local matrices (classes prefixed with {\tt Op}) and 
      data layout {\it schemas}.

\item Linear operators and elemental operations with them: assembly of a global 
      matrix (e.g. Jacobian), matrix-vector product, inversion, and calculation of the Schur complement.

\item Discrete PDEs: populate values in local matrices, add nonlinear 
coefficients, create specialized preconditioners, and impose special
boundary conditions. 
\end{enumerate}


\subsubsection{Op}
Class {\tt Op\_*} is a container of local matrices.
A series of such classes (e.g. {\tt Op\_Cell\_FaceCell} and {\tt Op\_Cell\_Schema}) handle data layout. 
The second word in the class name indicates the container size (the number of mesh cells here).
The third word specifies location of degrees of freedom: in cells and on faces in the first example;
specified by a complex schema in the second example.
These are really just structs of vectors of
dense matrices of doubles, and simply provide a type.
They are derived from the virtual class {\tt Op}.

A key concept of an {\tt Op} is the schema. 
The old design of the schema includes one enum representing the dofs associated
with the Operator's domain and range, and one enum for the contained size. 
This is a major limitation for implementing complex discretization schemes.
Also a single schema implies that the domain and range of the operator are the same.
The new design (which is backward compatible) includes two schemas that are also more 
detailed. A the first enum is replaced with the list of enums to represent various 
possible collections of degrees of freedom including non-standard degrees of freedom 
as as derivatives and moments.
Additional list specifies multiplicity of these degrees of freedom.
The second enum in the new schema specifies (as before) the geometric entity over which the local 
matrices are assembled.

Existence of two (simple and complex) schemas provides some flexibility for code development.
For instance, a developed could create surface matrices, and then assemble them into a 
subsurface matrix by introducing a new {\tt Op} class (for surface discretization) with a simple schema.
Alternatively, it could be done using the class {\tt Op\_Cell\_Schema} with a complex schema. 

The general schema makes it trivial to assemble a global matrix (e.g. in a coupled flow-energy system)
from sub-block operators.
Finally, the new schema supports rectangular matrices which is useful for saddle-point 
type systems.

{\tt Op\_*} works via a visitor pattern.
Matrix assembly, {\it Apply()}, application of boundary conditions, and symbolic assembly 
are implemented by the virtual class {\tt Operator} calling a dispatch to the 
virtual class {\tt Op}, which then dispatches back to the derived class {\tt Operator\_*} so that
type information of both the {\tt Operator\_*} (i.e. global matrix info) and 
the {\tt Op\_*} (i.e. local matrix info) are known.

A container of local matrices (i.e. instantiation of a {\tt Op\_*}) 
can be shared by multiple {\tt Operator\_*}. 
Sharing is indicated by the variable {\it ops\_properties}. 
In combination with {\it CopyShadowToMaster()} and {\it Rescale()},
a developer has a room for a variety of optimized implementations.
The key parameters have prefix {\rm OPERATOR\_PROPERTY} and described in file {\it Operators\_Defs.hh}.


\subsubsection{Operator}
An operator represents a map from linear space $X$ to linear space $Y$.
Typically, this map is a linear map; however, it can be used also to calculate
a nonlinear residual. 
The spaces $X$ and $Y$ are coded using class {\tt CompositeVectorSpace}.
A few concrete maps $X \to Y$ are already implemented in the code.

Typically the forward operator is applied using only local Ops.
The inverse operator typically requires assembling a matrix, which 
may represent the entire operator or may be only its Schur complement.

The class {\tt Operator} performs actions summarized in the second bullet above. 
Amanzi has a few derived classes such as {\tt Operator\_Cell}, {\tt Operator\_Node}, 
{\tt Operator\_FaceCellSff}, where the suffix {\tt \_X} indicates the specific map,
see class {{\tt Operator\_Schema} for a general map.
These classes are derived from the virtual class {\tt Operator} which stores a
schema and a pointer to a global operator.

Concrete maps use the old schema which is an integer variable.
Their are now superseded by the new flexible schema which is a class variable.
Each operator stores a list of containers of local matrices, more precisely 
a list of pointers to the variables of class {\tt Op}.

The only potentially confusing part is the use of the visitor pattern (i.e. double 
dispatch in this case) to resolve all types.  
For instance to assemble a matrix, we may use the following pseudocode

\begin{lstlisting}[language=C++]
// Operator
AssembleMatrix(Matrix A) {
  for each op {
    op->AssembleMatrix(this, Matrix A);
  }
}

virtual AssembleMatrixOp(Op_Cell_FaceCell& op) { 
  // throw error, not implemented
}

// Op
AssembleMatrix(Operator* global_op, Matrix& A) = 0;

// Op_Cell_FaceCell
AssembleMatrix(Operator* global_op, Matrix& A) {
  global_op->AssembleMatrixOp(*this, A);
}

// Operator_FaceCell
AssembleMatrixOp(Op_Cell_FaceCell& op, Matrix& A) {
  // This method now know both local schema and the matrix's dofs, 
  // and assembles the face+cell local matrices into the matrix.
}
\end{lstlisting}

The reason for the double dispatch is to get the types specifically
without a ton of statements like this one "if (schema == schema1) 
\{ assemble one way \} else \{ assemble another way\}".


\subsubsection{PDE}
A "discrete" PDE consists of (a) a single global operator, (2) an 
optional global assembled matrix, and (3) an un-ordered additive collection of 
lower-rank (or equal rank) local operators, hereafter called {\it ops}. 
During its construction, a PDE can grow by assimilating more {\it ops}. 
The global operator knows how (a) to perform the matrix-vector product, 
the corresponding function is called {\it Apply()}, and (b) to assemble {\it ops} into a global matrix.
Each {\tt PDE\_*} class knows how to apply boundary conditions and to create a preconditioner.

The classes {\tt PDE\_Diffusion}, {\tt PDE\_Advection},{\tt PDE\_Accumulation}, etc create 
operators of the specified type (for instance 
{\tt Operator\_FaceCell} or {\tt Operator\_Schema}), populate their values, and
apply boundary conditions.
They are in some sense physics based generalization of operators and may perform complex actions
such as an approximation of Newton correction terms.

A collection of PDEs that store a pointer to the same global operator form an additive PDE.
Application of boundary conditions is done independently by each PDE in this collection.
The result is gathered into a single right-hand side vector.


Discretization of a simple 
PDE (i.e. diffusion) is not done directly. 
Instead, a helper class that contains methods for creating and populating 
the {\it ops} within the {\tt Operator} is used. 
The helper class can be used to discretize a simple PDE, such as the diffusion equation.
A more complex PDE, such as the advection-diffusion equation, can be discretized 
by creating two "discrete" PDEs for diffusion and advection processes.


%%%%%%%%%%%%%%%%%%%%%%%%%%%%%%%%%%%%%%%%%%%%%%%%%%%%%%%%%%%%%%%%%%%%%%
\subsection{TreeOperator}
Class {\tt TreeOperator} is the block analogue of linear operators and 
provides a linear operator acting on a {\tt TreeVectorSpace}. 
In short, it is a matrix of operators.
Currently this structure is used for things like multiphase flows, 
thermal Richards, coupled matrix-fracture flow and transport, etc.


%%%%%%%%%%%%%%%%%%%%%%%%%%%%%%%%%%%%%%%%%%%%%%%%%%%%%%%%%%%%%%%%%%%%%%
\subsection{Linear solvers}
Native and third-party solvers are handled through a single factory and 
the uniform interface.
Direct and iterative solvers from Trilinos is a part of this factory.
Native re-implementation of some iterative solvers supplied by Trilinos
is due to lack of capabilities needed for subsurface solvers.
One example is the neccisety to perform at least one iteration even when
a norm of the linear residual is below the requested tolerance.


%%%%%%%%%%%%%%%%%%%%%%%%%%%%%%%%%%%%%%%%%%%%%%%%%%%%%%%%%%%%%%%%%%%%%%
\subsection{Nonlinear solvers}
A factory of nonlinear solvers includes sever solvers ranging from
the Newton method to inexact Newton methods to continuation methods.
The solvers are templated on classes {\tt Vector} and {\tt VectorSpace}.

The nonlinear Krylov accelerator solvers \cite{carlson1998design} implements
inexact Newton's method, where the correction 
equation of Newton's method is only approximately solved because the 
Jacobian matrix is approximated and/or the linear system is not solved exactly.  
Placed in the iteration loop, this black-box accelerator listens to the sequence
of inexact corrections and replaces them with accelerated corrections;
the resulting method is a type of accelerated inexact Newton method.
Note that an inexact Newton iteration is merely a standard fixed point iteration for
a preconditioned system, and so this accelerator is more generally
applicable to fixed point iterations.


