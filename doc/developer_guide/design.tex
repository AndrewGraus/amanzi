% =========================================================================
% -------------------------------------------------------------------------
% Design:
% -------------------------------------------
%
%  This is a good place to outline key objectives of this section.
%
% -------------------------------------------------------------------------

\section{Design of Amanzi}

The design principles described below are about the code modularity and extensibility.
We follow closely, but not exactly, the Google C++ codying style.
The titles of subsections below are mapped on the source code names
to easy navigation.


%%%%%%%%%%%%%%%%%%%%%%%%%%%%%%%%%%%%%%%%%%%%%%%%%%%%%%%%%%%%%%%%%%%%%%
\subsection{State}

State is a simple data manager. 
It allows process kernels to require, read, and write various fields.
It provides some data protection by providing both const and non-const data pointers for fields.
It provides some initialization capability -- this is where all independent fields can be 
initialized since independent variables are typically owned by the state, not by a ny process kernel.



%%%%%%%%%%%%%%%%%%%%%%%%%%%%%%%%%%%%%%%%%%%%%%%%%%%%%%%%%%%%%%%%%%%%%%
\subsection{PK and MPC PK}

PK stands for Process Kernel.
MPC stands for Multi-Process Coupler.
Each PK and MPC PK does little actual numerical work.
Instead, PK administrates discretization schemes, time integrators, and solvers. 
Each PK may represent a single equation (e.g. the Poisson equation for the Darcy flow) 
or system of strongly connected equations (e.g. the Navier-Stokes flow).

An MPC PK couples multiple physical processes which have their respected PKs.
An MPC PK may often be fully automated with no knowledge of the underlying PKs.
Since an MPC PK has the same interface as a PK, it is also a process kernel which
allows us to build a hierarchy of physical models with various degree of coupling
ranging from a weak coupling to an iterative coupling to a strong coupling.

Much of the work in a PK is delegated to field evaluators, which implement various 
physical and mathematical models, such as the equations of state. 
For these reason, it is approprite to call them variable evaluators.
The available variable eveluators are classified as follows:

\begin{enumerate}
\item Independent variable evaluators are the user-provided functions of spatial and temporal coordinates
      and has no dependencies.
      They could be used to compute boundary terms, source terms, and initial conditions. 
\item Primary variable evaluatora are related to the fields solved for within a PK.
      Examples are pressure and temperature fields.
      Typically these evaluators are used internally to track change in fields state and inform the 
      dependency tree about this.
\item Secondary variable evaluators are derived either from primary variable evaluators or other secondary variables. 
      There are two types of the secondary variable evaluators used to evaluate either a single or multiple variable.
      A model for a secondary variable can be anything from a constitutive relation to a discrete operator
      (apply a divergence operator to a velocity given a mesh and discretization) 
      to a summation operator (add the divergence of Darcy fluxes to a source term to determine the mass balance).
      Quite often, the secondary field/variable evaluators are created by high-level PKs during the setup phase 
      and inserted automatically in the list of evaluators. 
\end{enumerate}

The evaluator is much like a functor or function; it stores no actual data, only metadata and 
a few parameters or constants.
It accesses data using a data manager, which controls access for both read-only and read/write modes. 

All evaluators are stored in a dependency graph, which is a directed, acyclic graph (DAG) 
describing the functional relationship of each field in the state. 
End nodes in the dependency graph are either independent variables or primary variables. 
All other nodes in the graph are secondary variables.

The combination of a data manager and a dependency graph enables dynamic definition of each variable's model 
and data, and splits complex equations into manageable chunks. 
It also allows lazy evaluation, where nodes in the graph are updated (re-calculated) only if their dependencies
have changed, resulting in a managed, automated evaluation process with fewer bugs and inefficiencies.

For more details, we refer to \cite{coon2016managing}.


%%%%%%%%%%%%%%%%%%%%%%%%%%%%%%%%%%%%%%%%%%%%%%%%%%%%%%%%%%%%%%%%%%%%%%
\subsection{CompositeVector and TreeVector}

Class {\tt CompositeVector} is an implementation of an improved
{\tt Epetra\_MultiVector} which spans multiple components and knows how to
communicate itself.
A composite vector is a collection of vectors defined on a common mesh and
communicator. 
Each vector, or component, has a name (typically, a mesh entity)
and a number of degrees of freedom.  

Ghost cell updates are managed by the {\tt CompositeVector}. 
The design of this pattern is prompted by two things:
\begin{itemize}
\item The need for updated ghost cell information is typically known by the
      user just prior to being used, not just after the non-ghost values are
      updated.
\item Occasionally multiple functions need ghost values, but no changes to
      owned data have been made between these functions.  However, it is not
      always possible for the second call to know, for certain, that the first
      call did the communication.  Versatility means many code paths may be
      followed.
\end{itemize}

\subsubsection{Future work: reduce number of parallel communications}
To avoid unncessary parallel communication the following algorithm were implemented
but are not active now.
This may change in the future.

Each time the values of the vector are changed, flags are marked to
record that the ghost values are stale.
Each time ghost cells are needed, that flag is checked and communication
is done, if needed.
Keeping these flags correct is therefore critical. 
To do this, access to vectors must follow the rigid pattern.
The following modifications tag the flag:

\begin{enumerate}
\item Any of the usual {\it PutScalar()}, {\it Apply()}, etc methods.
\item Non-const calls of {\tt ViewComponent()}.
\item Call of {\it GatherMasterToGhosted()} and {\it ChangedValues()}.
\item {\it Scatter()} called in non-INSERT mode.
\end{enumerate}

There exist known ways to break this paradigm. 
One is to store a non-const pointer to the underlying {\tt Epetra\_MultiVector}.
The fix is simple as this: NEVER store a pointer to the underlying data, 
just keep pointers to the CompositeVector itself.

The other one is when one grabs a non-const pointer, call {\it Scatter()}, then 
change the values of the local data.  
This is the nasty one, because it is both subtle and reasonable usage.
When you access a non-const pointer, the data is flagged as changed.
Then you call {\it Scatter()}, the data is flagged as unchanged.
Then you change the data from your old non-const pointer, and the data is changed, but not flagged.
The first fix is to always call {\it ViewComponent()} after {\it Scatter()} and before changing values.
Another way to protect yourself is to put non-const references in their own scope.
For instance, the following practice is encourage:
\begin{lstlisting}
CompositeVector my_cv;
{ // unnamed scope for my_vec
  Epetra_MultiVector& my_vec = *my_cv.ViewComponent("cell", false);
  my_vec[0][0] = 12;
} // close scope of my_vec

my_cv.ScatterMasterToGhosted()

// Reference to my_vec is now gone, so we cannot use it and screw things up!

{ // unnamed scope for my_vec
  // This is now safe!
  Epetra_MultiVector& my_vec = *my_cv.ViewComponent("cell", true);
  my_vec[0][0] = my_vec[0][ghost_index] + ...
} // close scope of my_vec
\end{lstlisting}

the final way to break the parallel mashinery is use {\it const\_cast()} and 
then change the values.
Const-correctness is your friend. Keep your PKs const-correct, and you will never have this problem.

Note that non-INSERT modes of scatter are always done, and also always tag as changed.  
This is because subsequent calls with different modes would break the code.


%%%%%%%%%%%%%%%%%%%%%%%%%%%%%%%%%%%%%%%%%%%%%%%%%%%%%%%%%%%%%%%%%%%%%%
\subsection{Operators and TreeOperators}

Again, mimicks the structure of PKs and MPC PKs

The idea behind the desing of Amanzi operators is to separate three 
functionalities that are frequently placed in a single class in other
C++ packages.

\begin{enumerate}
\item Containers of local matrices ({\tt Ops}) and data layout {\tt schemas}.

\item Linear operators and elemental operations with them: assembly of a global 
      matrix, matrix-vector product, inversion, and calculation of the Schur complement.

\item Discrete PDEs: populate values in local matrices, add nonlinear 
coefficients, create specialized preconditioners, and impose special
boundary conditions. 
\end{enumerate}

\subsubsection{Op}
Op is a container of local matrices.
A series of {\tt Op\_BASE\_DOFS} classes (such as {\tt Op\_Cell\_FaceCell} and 
{\tt Op\_Cell\_Schema}) handle data layout (item \#1). 
These are really just structs of vectors of
dense matrices of doubles, and simply provide a type.
They are derived from the virtual class {\tt Op}.

A key concept of an {\tt Op} is the schema. 
A schema includes at least one enum representing the dofs associated
with the Operator's domain and range. A single schema is a major limitation 
since it implies $X=Y$.
The new design (backward compatible) includes two schemas that are more 
detailed. A list of enums allows us to represent various collections of 
degrees of freedom including derivatives and vector components.
A schema includes also information on the base entity on which the local 
matrix lives.

This desing enables a few things for future code development.
For instance, it should make creating surface matrices, and then assembling into a 
subsurface matrix doable by introducing a new {\tt Op} class with a simple schema
or using class {\tt Op\_Cell\_Schema} with a complex schema. 
It also makes it trivial to assemble the global operator into a bigger, containing 
matrix (i.e. energy + flow) as any of the four sub-blocks.
Finally, the new schema support rectangular matrices useful for copling with 
Stokes-type systems.

Ops work via a visitor pattern.
Matrix assembly, "Apply", apllication of boundary conditions, and symbolic assembly 
are implemented by the virtual class {\tt Operator} calling a dispatch to the 
virtual class {\tt Op}, which then dispatches back to the derived class Operator so that
type information of both the Operator (i.e. global matrix info) and 
the Op (i.e. local matrix info) are known.

Ops can be shared by Operators. 
In combination with {\it CopyShadowToMaster()} and {\it Rescale()},
the developer has a room for a variaty of optimized implementations.
The key variable is ops\_properties. The key parameters are 
OPERATOR\_PROPERTY and described in Operators\_Defs.hh.


\subsubsection{Operator}
An Operator represents a map from linear space $X$ to linear space $Y$.
Typically, this map is a linear map; however, it can be used also to calculate
a nonlinear residual. 
The spaces $X$ and $Y$ are described by class {\tt CompositeVector}. 
A few maps $X \to Y$ are supported now.

Typically the forward operator is applied using only local Ops.
The inverse operator typically requires assembling a matrix, which 
may represent the entire operator or may be only its Schur complement.

The class {\tt Operator} performes actions summarized in item \#2. There exists
a series of inheriting classes such as {\tt Operator\_Cell}, {\tt Operator\_Schema}, 
{\tt Operator\_FaceCellSff}, where the suffix {\tt \_X} indicates the map/matrix.
Explicit maps are superceded by the flexible new schema.
They are derived from the virtual class {\tt Operator} which stores a local 
and global schemas. 
One Operator is marked as the global operator. It stores a list of Op classes
with compatible (equal or smaller) schemas.

The only potentially confusing part is the use of the visitor pattern (i.e. double 
dispatch in this case) to resolve all types.  
For instance to assemble a matrix, we may use the following pseudocode

\begin{lstlisting}[language=C++]
// Operator
AssembleMatrix(Matrix A) {
  for each op {
    op->AssembleMatrix(this, Matrix A);
  }
}

virtual AssembleMatrixOp(Op_Cell_FaceCell& op) { 
  // throw error, not implemented
}

// Op
AssembleMatrix(Operator* global_op, Matrix& A) = 0;

// Op_Cell_FaceCell
AssembleMatrix(Operator* global_op, Matrix& A) {
  global_op->AssembleMatrixOp(*this, A);
}

// Operator_FaceCell
AssembleMatrixOp(Op_Cell_FaceCell& op, Matrix& A) {
  // This method now know both local schema and the matrix's dofs, 
  // and assembles the face+cell local matrices into the matrix.
}
\end{lstlisting}

The reason for the double dispatch is to get the types specifically
without a ton of statements like this one "if (schema | XX \&\& schema | YY) 
\{ assemble one way \} else \{ assemble another way\}".


\subsubsection{TreeOperator}


\subsubsection{PDE}
An discrete PDE consists of (a) a single global operatorr, (2) an 
optional assembled matrix, and (3) an un-ordered additive collection of 
lower-rank (or equal) local operators, here called Ops. 
During its construction, a PDE can grow by assimilating more Ops. 
The global Operator knows how to peform the method "Apply" and assemble 
all of its local Ops.
The PDE knows how to apply boundary conditions and to create a preconditioner.

The classes {\tt Diffusion}, {\tt Advection}, {\tt Elasticity},  
and {\tt Accumulation} in item \#3 create operators of the right type (for instance 
{\tt Operator\_FaceCell} or {\tt Operator\_Schema}), populate their values, and stick 
them in a global operator.
These are physics based operators and perform complex operations such as approximation
of Newton correction terms.

Application of boundary conditions is driven by a discrete PDEs class. 
Hence, each PDE object contributing to the global PDE object must apply
boundary conditions itself.


\subsubsection{Note on the discretization of PDEs}
Discretization of a simple 
PDE (i.e. diffusion) is not constructed directly. 
Instead, a helper class that contains methods for creating and populating 
the Ops within the Operator is created. This helper class can create the
appropriate discretization itself. More complex PDES, for instance the
advection-diffusion, can be discretized by creating a global Operator that 
is the union of all dofs requirements, and then passing this Operator
into the helper's constructor. When this is done, the helper simply 
checks to make sure the Operator contains the necessary dofs and
adds local Ops to the global Operator's list of Ops.


\subsection{Linear solvers}

Native and third-party solvers are handle through a single factory and 
uniform interface.



