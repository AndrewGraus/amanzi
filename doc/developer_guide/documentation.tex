% =========================================================================
% -------------------------------------------------------------------------
% Documentation:
% -------------------------------------------
%
%  This is a good place to outline key objectives of this section.
%
% -------------------------------------------------------------------------

\section{Documentation}

\subsection{User Guide}
The description of each test in the user guide uses the structures test format
and consists of a few sections.

\begin{enumerate}
\item \underline{Introduction} describes purpose of the test.
\item \underline{Problem Specification} describes the physical and mathematical model 
      with the level of details sufficient to reproduce the results.
\item \underline{Results and Comparison} summarizes numerical results in a form of plots
      and tables (e.g., drawdown curves). Comparison with analytic data, data produced 
      by other codes, or data published elsewhere is strongly encouraged.
\item \underline{References}.
\item \underline{About} collects technical information for developers that include 
      location of the files, names of the developers, names of critical files (XML input,
      Exodus mesh, analytic data, etc), names of output files.
\item \underline{Status} describes status of the test and required future work.
\end{enumerate}


\subsection{Native Spec}
This is a continuously evolving specification format used by the code developers. 
Its main purpose is to develop and test new capabilities without disruption of end-users.
The documentation is in the form of a structured text, see {\tt doc/input\_spec/AmanziNativeSpecV8.rst}.



