\section{Synthetic example}
The purpose of this section is to provide step by step instructions into 
solving square systems using Amanzi.
A square systems are those that can be written as
$$
  \mbox{Find~} u \in V \colon \quad
  a(u, v) = f(v) \quad \forall  v \in V.
$$

\subsection{Glossary}

\subsection{Defining your PDE class}
The PDE class defined must be a derived class of PDE HelperDiscretization,
this immediately gives new class access to two important varibles, a global operator and a container for 
local matrices and a series of useful routines to apply boundary conditions, assemble global systems, etc.
This all comes with a caviat: you must define a member function called UpdateMatrices which is usually
used to populate the local matrices, failure to do so will result in an abstract class with no possibility for
instantiation.
The header for a class to solve the Poisson equation in second order form will look like

\subsection{Populating local matrices and defining the global operator}
The first step in populating the local matrices is to define a schema. There should be one schema for the test space and one for
the trial space, in the case of square systems the same can be used for both. Schemas define the different
aspects of a variable and its discretization. Schemas require two inputs: the base and and an item. The base
describes what type of assembly is required for this variable the choices include cells, faces, edges and nodes
all part of the AmanziMesh namespace. Selecting, for example, faces as the base will imply that the local
matrices are associated with the faces of the mesh. Moreover, item defines the type of degrees of freedom
that are used to discretize the variable in question. Items require three inputs: a part of the topology of the
mesh like a node or an edge which defines where the degrees of freedom are places, the type of quantity
the degree of freedom, whether it is scalar of vector valued and the number of degrees of freedom of this
type. For a classic finite element method to solve the Poisson equation the definition of its schema will look
something like


\subsection{Creating a mesh}


\subsection{Adding boundary conditions} 
The class PDE HelperDiscretization has some built in features to
impose boundary conditions but in order to access them we need to define the object BCs which takes
as creation arguments the mesh, where the degrees of freedom will be places and the type of degree of
freedom. Moreover, we must also populate the two class variables, the bc model which defines what type of
boundary condition we want to prescribe and the bc value which precise value of such boundary condition.
For example,


\subsection{Assembly and imposing the boundary conditions}


\subsection{The linear solve}
In order to apply a linear solver we must initialize a vector for the solution and
initialize the preconditioner. The linear solve is templated to fit the different types of scenarios where linear
solves are necessary.



