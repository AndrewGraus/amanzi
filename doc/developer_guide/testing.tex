% -------------------------------------------------------------------------
%  This is a good place to outline testing strategiesx in Amanzi
% -------------------------------------------------------------------------

%%%%%%%%%%%%%%%%%%%%%%%%%%%%%%%%%%%%%%%%%%%%%%%%%%%%%%%%%%%%%%%%%%%
\section{Code development cycle}
Development of a new capability consists of several steps that are summarized below.
Some steps can be skipped during a casual work cycle of code support, bug fixes, and minor improvements. 

\begin{itemize}
\item Create a new github development branch.
\item Create github ticket or multiple tickets that summarize and stage the development process.
\item Implement numerical algorithms and add them to an Amanzi library.
\item Write unit tests for the new code.
\item Integrate new functionality into other algorithms.
\item Write integrated unit tests as needed.
\item If implemented algorithms take control parameters from an XML file, 
      document these parameters.
\item Test new capability and add a benchmark or verification test to the
      user guide.
\item Create a pull request to inform team members about the new capability and 
      to collect miscallenous feedback.
\item Merge the development branch into the master branch.
\end{itemize}
\clearpage



%%%%%%%%%%%%%%%%%%%%%%%%%%%%%%%%%%%%%%%%%%%%%%%%%%%%%%%%%%%%%%%%%%%
\section{Testing}
Testing is a cornerstone of modern software development. In the form of Test-Driven
Development, it is useful for providing feedback in the design process. In other 
forms, it is essential for preventing the project from descending into chaos, and 
controlling the cost of software maintenance. In this section we describe the 
various forms of testing used to certify that Amanzi works properly, in order of 
increasing scope.

% Describe testing layout in directories?


\subsection{Unit Testing}
Each individual software component should have a defined set of assumptions under 
which it operates, and a set of behaviors and corresponding certifications on 
what it produces. These assumptions and behaviors are ideally articulated in the 
documentation of the component, but they should also be tested independently as part
of the implementation process. A test of an individual component's assumptions and 
behaviors is called a {\em unit test}. A unit test provides a set of PASS/FAIL tests for 
each function, method, and attribute in a software component.

Some Amanzi's tests are integrated tests that fill a huge gap between short unit tests
and long benchmark tests.
At the moment they are also called unit tests. 

% Talk about Amanzi's unit testing here.


\subsection{Verification and Benchmark Testing}
The various algorithms we use in Amanzi have to be tested on the basic subsurface 
problems that are relevant to our charter, and compared against other codes to 
weigh the costs and benefits of our choices against existing approaches. 

A {\em verification test} consists of a simulation run with a given input describing 
a problem that has a known solution, a characterization of the quality of the 
solution, and a PASS or FAIL result based on the quality of that solution measured 
against some threshold. 

% Describe verification testing here.

A {\em benchmark test} is a simulation run with a given input whose output is 
compared to the output of one or more other codes. All codes must have inputs that 
describe the same ``benchmark problem." The differences between the codes can be 
evaluated visually and/or with numerical metrics. Numerical metrics allow benchmark 
tests to have PASS/FAIL results, whereas a visual inspection test requires an 
expert for evaluation, so the former are preferred where practical.

% Describe benchmark testing here.


\subsection{Regression Testing}
A {\em regression test} is a simulation-based PASS/FAIL test similar to a 
verification test, and is typically part of a large suite of tests that are run 
automatically and periodically to ensure that bugs and errors have not been 
introduced into Amanzi during code development. We provide a couple of tools for 
constructing PASS/FAIL tests that can be used to monitor bugs and regressions. In 
particular, we support two types of regression tests: {\em smoke tests} and 
{\em comparison tests}.

\subsubsection{Smoke tests}
A smoke test simply runs an Amanzi simulation with a given input, PASSes if 
the simulation runs to completion, and FAILs otherwise. A smoke test can be created
(and added to Amanzi's regression test suite) by calling the following CMake command 
inside of a CMakeLists.txt file in a testing directory:

\begin{lstlisting}
ADD_AMANZI_SMOKE_TEST(<test_name> 
                      INPUT file.xml
                      [FILES file1;file2;...;fileN]
                      [PARALLEL] 
                      [NPROCS procs1 ... ]
                      [MPI_EXEC_ARGS arg1 ... ])
\end{lstlisting}

Arguments:
\begin{itemize}
\item \verb|test_name|: the name given to the comparison test 
\item \verb|INPUT| (required): This (required) keyword defines an Amanzi XML input file that will 
      be run.
\item \verb|FILES| (optional): A list of any additional files that the test needs in order 
      to run in its directory/environment. These files will be copied from the source 
      directory to the run directory.
\item \verb|PARALLEL| (optional): The presence of this keyword signifies that this is 
      a parallel job. This is also implied by an NPROCS value > 1
\item \verb|NPROCS| (optional): This keyword starts a list of the number of processors to
      run the test on, and defaults to 1.
\item \verb|MPI_EXEC_ARGS| (optional): This keyword denotes extra arguments to give to
      MPI. It is ignored for serial tests.
\end{itemize}

\subsubsection{Comparison tests}

A comparison test runs an Amanzi simulation with a given input, and then compares 
a field or an observation from that simulation to that in the specified reference 
file, PASSing if the L2 norm of the difference in the simulation and reference 
values falls below the given tolerance. One can add a comparison test to the 
Amanzi regression test suite by calling the following CMake command inside of a 
CMakeLists.txt file within a testing directory:

\begin{lstlisting}
ADD_AMANZI_COMPARISON_TEST(<test_name> 
                           INPUT file.xml
                           REFERENCE reference
                           [FILES file1;file2;...;fileN]
                           ABSOLUTE_TOLERANCE tolerance
                           RELATIVE_TOLERANCE tolerance
                           [FIELD field_name]
                           [OBSERVATION observation_name]
                           [PARALLEL] 
                           [NPROCS procs1 ... ]
                           [MPI_EXEC_ARGS arg1 ... ])
\end{lstlisting}

Arguments:
\begin{itemize}
\item \verb|test_name|: the name given to the comparison test 
\item \verb|INPUT| (required): This (required) keyword defines an Amanzi XML input file that will 
      be run.
\item \verb|REFERENCE| The name of the file containing reference data to which 
      the simulation output will be compared.
\item \verb|TOLERANCE| (required): This specifies the maximum L2 error norm that can be 
      measured for a successful testing outcome.
\item \verb|FILES| (optional): A list of any additional files that the test needs in order 
      to run in its directory/environment. These files will be copied from the source 
      directory to the run directory.
\item \verb|FIELD| (required if OBSERVATION not given): The name of the field in Amanzi that 
      will be compared to its reference value for this test.
\item \verb|OBSERVATION| (required if FIELD not given): The name of the observation in the Amanzi 
      input that will be compared to its reference value for this test.
\item \verb|PARALLEL| (optional): The presence of this keyword signifies that this is 
      a parallel job. This is also implied by an NPROCS value > 1
\item \verb|NPROCS| (optional): This keyword starts a list of the number of processors to
      run the test on, and defaults to 1.
\item \verb|MPI_EXEC_ARGS| (optional): This keyword denotes extra arguments to give to
      MPI. It is ignored for serial tests.
\end{itemize}

