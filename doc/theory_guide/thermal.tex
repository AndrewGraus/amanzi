% =========================================================================
% -------------------------------------------------------------------------
% Thermal:
% -------------------------------------------
%
%  Non-isothermal (heat)
%
% -------------------------------------------------------------------------

\section{Thermal Flow Processes}

\label{sec:thermal-processes}

\subsection{Overview}

Heat flow and thermal conduction is an important aspect of many geochemical systems 
affecting chemical processes through changes in equilibrium and kinetic rate constants. 
Non-isothermal conditions can also result in buoyancy driven flow leading to convection cells 
and causing fingering phenomena due to differences in density.




%\subsection{Data Needs}

Equations of state for fluid density, internal energy and/or enthalpy are needed 
in addition to heat capacity and thermal conductivity of the porous medium. 
Often the fluid properties for a complex mixture are unknown and the pure phase end member properties are used. 


%---------------------------------------------------------------------------------------------
% Conservation Equations - Energy.
%---------------------------------------------------------------------------------------------
\subsection{Model Equations}

As heat flow is coupled to the Darcy flux, the heat equation itself is coupled to the flow equation 
as well as reactive transport equations through heat generated by chemical reactions. 
Conversely the flow and reactive transport equations are coupled to the heat equation 
through the temperature dependence of fluid properties such as density, viscosity, internal energy and enthalpy, 
and equilibrium thermodynamic and kinetic rate constants.


\subsubsection{Partially saturated flow with water vapors}


In the case of isothermal Richards equations it was possible to neglect water vapors, as 
presumably their concentration was the same everywhere. 
In the case when temperature variations cannot be neglected the effects of water vapors could be significant.

The Richards equation is derived from the conservation of liquid mass equation.
Thus, changing vapor concentrations introduce an additional term 
$$
  -\bnabla \cdot \left( \K_g \bnabla \frac{p_v}{p_g} \right) 
$$
into \eqref{eq:Darcy}, where 
$p_v$ and $p_g$ [$\upa$] are the vapor and the gas pressures, respectively,
and
$\K_g$ is the effective diffusion coefficient of the water vapors.
The gas pressure is presumed to be equal to the atmospheric one,
while water vapor pressure is derived from the equilibrium of liquid and gas phases:
\[
  p_v = P_\text{sat} (T) \exp\left(\frac{P_\text{cgl}}{\eta_l RT}\right),
\] 
where 
$P_\text{sat}$ is the saturated vapor pressure,
$P_\text{cgl}$ is the capilary gas-liquid pressure, and 
$R$ is the ideal gas constant.  

The Richards equation with gas vapors takes the form
\begin{subequations}\label{eq:Richards with vapors}
\begin{align}
  \frac{\partial \theta(p_l)}{\partial t} 
  &= 
  -\bnabla \cdot (\eta_l \bq_l)
  -\bnabla \cdot \left( \K_g \bnabla \frac{p_v}{p_g} \right) 
  + Q,
  \\
  \bq_l 
  &= 
  -\frac{\K k_{rl}}{\mu_l} (\bnabla p_l - \rho_l \bg),
\end{align}
\end{subequations}
where 
$\eta_l$ is the molar liquid density,
$Q$ is source or sink term [$\ukg \ucdot \um^{-3} \ucdot \us^{-1}$],
$\K$ is absolute permeability tensor,
$\mu_l$ is liquid viscosity,
$\rho_l$ is liquid density, and
$k_{rl}$ is relative permeability.
The total volumetric water content $\theta$ has to be modified to include 
the vapors
$$
  \theta = \phi \eta_l\, s_l + \phi (1-\eta_l) X_l,
$$
where $X_l$ is molar fraction of water vapors.

The effective diffusion coeficient of water vapor is given by 
\[
  \K_g = \phi g_g \tau_g \eta_g D_g,
\]
where $s_g$ is gas saturation, 
$\tau_g$ is the gas tortuosity,
$\eta_g$ is the molar density of gas, and 
$D_g$ is the gas diffusion coefficient.
The later, based on the TOUGHT2 model, is 
\[
  D_g = D_\text{ref} \frac{P_\text{ref}}{p_g} \left(\frac{T}{T_\text{ref}}\right)^\alpha,
\]
where 
$D_\text{ref} = 2.14\times 10^{-5}$,
$P_\text{ref}$ is the atmospheric pressure,
$T_\text{ref} = 273.15$,
$\alpha=1.8$.
Finally, we need a model for the gas tortuosity
for which we use \citet{millington1961permeability} 
model
\[
  \tau_g = \phi^\beta, s_g^\gamma,
  \qquad \beta = 1/3, \quad \gamma = 7/3.
\]




\subsubsection{Energy Conservation Equation}
\label{sec:Energy}

The Richards equation with water vapor is coupled with the conservation of energy equation.
One form of such equation in a porous medium with porosity $\phi$ is given by
\begin{align}
\label{eq:energy_balance}
  \frac{\partial}{\partial t} \Big[ 
  \phi \rho_l u_l s_l
  + 
  (1-\phi )\rho_r u_r 
  \Big] \ 
  + \ &
  \boldsymbol{\nabla} \cdot \left[ 
  -\frac{\K k_{rl} }{\mu _l } (\boldsymbol{\nabla} p_l 
  -
  \rho _{\alpha} g \boldsymbol{e}_z)\rho _l h_l 
  - 
  Q_T 
  \right] 
  \eq 
  Q_e,
\end{align}
where
$\rho_{r}$ and $\rho_{l}$ are the density, 
$u_{r}$ and $u_l$ are the internal energy 
of the rock and the liquid phases, respectively;
$\K$ is the absolute permeability tensor,
$k_{rl}$ is the relative permeability coefficient,
$s_l$ is the liquid saturation,
$\mu_l$ is fluid viscosity,
$h_l$ is the enthalpy of the liquid phase, 
$Q_T$ represents thermal conduction and radiation, and 
$Q_{e} $ is a source or sink of energy. 
Note that the energy consumed/produced by chemical reactions is included in this last term. 

The primary variables in \eqref{eq:energy_balance} are the pressure $p_l$ 
and the temperature $T$. 
The internal energies $u_r$ and $u_l$ are both functions of temperature.  
Assuming that the internal energy $u_r$ and the thermal conduction $Q_T$
have linear dependence on temperature,  
$$
  u_r = c_r T \qquad \text{and}\qquad Q_T=\kappa T,
$$
the equation \eqref{eq:energy_balance} takes the form 
\EQ
  \frac{\partial}{\partial t} 
  \Big[ \phi s_l \rho_l u_l 
  + 
  (1-\phi) \rho_r c_r T \Big] 
  + 
  \boldsymbol{\nabla}\cdot \Big[
  \bq_l \rho_l h_l 
  - 
  \kappa\boldsymbol{\nabla} T\Big] 
  \eq 
  Q_e,
\EN
where 
$T$ refers to temperature, 
$\bq_l$ is the Darcy velocity 
\EQ
  \bq_l 
  =  
  -\frac{\K k_{rl} }{\mu _l } 
  \left(
  \boldsymbol{\nabla} p_l 
  -
  \rho _{\alpha} g \boldsymbol{e}_z
  \right), 
\EN
the coefficient $\kappa$ denotes the thermal conductivity of the medium and 
$c_r$ refers to the specific heat of the porous medium. 
The internal energy and enthalpy are related by the equation
\EQ
  u_l \eq h_l -\frac{p_l}{\rho_l}.
\EN
Thermal conductivity if often described by the phenomenological relation given by \citet{somerton1974high}
\EQ\label{cond} 
  \kappa \eq \kappa_{\rm dry} + \sqrt{s_l^{}} (\kappa_{\rm sat} - \kappa_{\rm dry}), 
\EN 
where $\kappa_{\rm dry}$ and $\kappa_{\rm sat}$ are dry and fully saturated rock thermal conductivities, 
and $s_l$ denotes the saturation state of the liquid. 

\paragraph{Evapotranspiration. } 
%
Evapotranspiration models should include all the processes that convert water from the aqueous phase into water in the gaseous phase, 
i.e., water vapor.  
This should also account for evaporation from soil and plant surfaces and plant transpiration and include options 
where these components vary wiith soil properties and structure of the plant canopy.

\paragraph{Equation of State.}
%
For multicomponent system, equation of state (EOS) data is required for water and all the NAPL components.  
Typically these are EOS for pure substances that are combined for mixtures.  
A basic EOS relates density to pressure and temperature.  
The form for the EOS is typically cubic such as the Soave-Redlich-Kwong (SRK) or the
Peng-Robinson (PR) models.  
Tabular models are also in common use.

Mixture thermodynamics is used to combine pure phase EOS's for the application.  
The EPA lists over 80 potential NAPL components
(volatile organic compounds or VOCs) that can cause groundwater
contamination and has data bases with at least some properties for these
contaminants.


\paragraph{Mixture Internal Energy and Enthalpy.} 
%
These will generally follow the simple additive rule based of
component values and mass fraction. Consideration must be also given
to heats of solution.



\subsection{Boundary Conditions}

The boundary conditions for the thermal flow 
\eqref{eq:Richards with vapors}-\eqref{eq:energy_balance}
consist of the conditions for each of the equations.
%the Richards equation with vapor
%<F6>\eqref{eq:Richards with vapors} and those for the conservation of energy equation \eqref{eq:energy_balance}.
The choice of the boundary conditions for \eqref{eq:Richards with vapors} are identical to those of Isothermal equaiton.
We repeat those equations here for the sake of self-contained presentation.
The boundary conditions for conservation of energy equation \eqref{eq:energy_balance}
may take the form of specified temperature or heat flux including zero temperature gradient. 
Initial conditions include specifying the temperature over the computational domain 
such as a constant value or derived from the geothermal gradient.


\paragraph{Specified temperature boundary conditions.}
These are Dirichlet type conditions on the temperature saying 
that at the boundary $\Gamma_{TD}$
\begin{align}
  T(\bx,t) = T_b(\bx,t), \quad \bx\in \Gamma_{TD}, 
\end{align}
where the temperature $T_b(\bx,t)$ is given.

\paragraph{Specified heat flux boundary conditions.}
These are Neumann type boundary conditiosn on the temperature saying that
at the boundary $\Gamma_{TN}$
\begin{align}
  \partial_{\bn} T(\bx,t) = \partial_{\bn} T_b(\bx,t), \quad \bx\in \Gamma_{TD}, 
\end{align}
where the temperature $\partial_{\bn} T_b(\bx,t)$ is given.





\paragraph{Boundary of prescribed pressure or head.}
This involves the specification of a fixed pressure or hydrostatic head on boundary $\Gamma_{lD}$.
For instance, a boundary of this kind occurs whenever the flow domain is adjacent to a body of open water.
Segments A-B and E-F in Fig.~\ref{fig:bc_flow} are examples of a boundary of prescribed potential.
The pressure or head boundary conditions are given functions, e.g.
\begin{equation}
\label{eq:BC pressure Darcy PS thermal}
  p_l(\bx,t) = p_{b}(\bx,t), \quad \bx \in \Gamma_{lD}. 
\end{equation}

\paragraph{Boundary of prescribed flux.}
This involves the specification of the flux normal to the boundary $\Gamma_{lN}$
(see segment C-D in Figure \ref{fig:bc_flow}):
\begin{equation}
\label{eq:BC flux Darcy PS thermal}
  \bq_l\cdot \bn = q_{b}(\bx,t), \quad \bx \in \Gamma_{lN},
\end{equation}
where $q_b$ [$\um\ucdot\us^{-1}$] is the given boundary flux. 
For infiltration at the top horizontal surface, it equals to the Darcy velocity
and referred to as the infiltration velocity.

\paragraph{Semipervious boundary (or mixed boundary condition).}
This boundary condition is more complicated than the first two as it involves a case 
in which local conditions within the computational domain influence the flux in or 
out of the domain.
This type of boundary occurs when the porous medium domain is in contact with 
a body of water continuum (or another porous medium domain, see for instance segments 
A-B and E-F in Fig.~\ref{fig:bc_flow}), however, a relatively thin semipervious layer 
separates the two domains:
\begin{equation}
\label{eq:BC semipervious Darcy PS thermal}
  \bq_l \cdot \bn = I\, \left(p (\bx,t) - p_{b}(\bx,t) \right), \quad \bx \in \Gamma_R.
\end{equation}
where $I$ is an impedance and $p_{b}(\bx,t)$ is the given external pressure.

\paragraph{Seepage face.}
As is shown in Fig.~\ref{fig:bc_flow} (see segments B-C and D-E), seepage face (or surface) 
is always present when a phreatic surface ends at the down-stream external boundary of flow domain.
In this case the phreatic surface is tangent to the boundary of the porous medium at points C and D.
Along a seepage surface, water emerges from the flow domain, trickling downward to the adjacent body of water.

A seepage surface is defined as the boundary 
where water leaves the ground surface and then continues to flow in a thin film along its surface.
Being exposed to the atmosphere, the pressure along the seepage face is equal to the atmospheric pressure 
(i.e. capillary pressure $p_{c}=0$). 

The geometry of the seepage face is known (as it coincides with the boundary of the porous medium), 
except for its limit (points C and D in Figure \ref{fig:bc_flow}) 
which is also lying on the (a priori) unknown phreatic surface.
The location of this point is, therefore, part of the required solution.



