\subsection{Aqueous Activity Coefficients}   \label{sec:AqueousActivityCoefficients}

\subsubsection{Overview}

This Toolset
%~\todo{are we capitalizing toolset?  is toolset consistent with other documents? -Williamson} 
includes models for thermodynamic activity coefficients
in aqueous solutions. Multiple models, each with its own set of
parameters and limitations, will be provided. In general, the toolset
user must choose one such model to use in a given modeling
application. In setting up to run the application, the user must
ensure that a matching database with the requisite model-specific
parameters is provided to support the run.

A key solution parameter associated with aqueous species activity
coefficients is the ionic strength, defined as
\begin{equation}
  \label{eq:IonicStrength}
\bar{I} \eq \frac{1}{2} \sum _{i} m_{i} z_{i}^{2}.
\end{equation}
Here $m_{i}$ denotes the molal concentration (molality) of the
$i^{th}$ solute species and $z_{i}$ denotes its electrical charge
number. Activity coefficients of charged solute species include a
functional dependence on the ionic strength. The exact nature of this
dependence depends on the specific model.

Activity coefficient models can be classified as to the upper limit of
ionic strength to which a given model provides generally satisfactory
results. For the most part, there are two kinds of such models. Low
ionic strength models are generally usable up to an ionic strength of
more or less 1 molal. Examples include the Davies equation and the
B-dot equation. These models are based on simple equations and have a
relatively small number of associated parameters. High ionic strength
models are usable to very high ionic strength ($>$20 molal).
The highest
values of ionic strength normally seen are limited by the solubilities
of highly soluble salts, such as calcium chloride and calcium nitrate.
Examples of high ionic strength models include Pitzer's equations and
Extended UNIQUAC. High ionic strength models have more complex
equations and require substantially more parameters than low ionic
strength models. They are most likely to be applied to problems in
which the ionic strength is higher than 1 molal. At low ionic
strength, it is generally preferable to use a low ionic strength
model, because the number of required parameters is smaller. In many
instances, there will be sufficient supporting data available to
support the use of low ionic strength models for large numbers of
chemical components and species, but that may not be the case for the
high ionic strength models. There are activity coefficient models that
extend to intermediate ionic strength (4-6 molal). The NEA SIT
model is an example. These models tend to be intermediate in equation
complexity and number of required parameters. They have not received
much attention to date in modeling geochemically complex systems.

It is noted that low ionic strength models are sufficient for many EM
applications. Hanford tanks and the WIPP site pose notable exceptions, requiring the
use of high ionic strength models. There may be other instances in
which the use of low ionic strength models may be inappropriate. 

There are two kinds of activity coefficients that a model should be
able to provide. The first is the molal activity coefficient of a
solute species (denoted by $\gamma _{i}$). This is subsequently used
to compute the thermodynamic activity of the corresponding species
(denoted by $a_{i}$). The activity of a species is obtained by
multiplying the molality (molal concentration) of the species (denoted
by $m_{i}$) by the molal activity coefficient:
%
\begin{equation}
  a_{i} =m_{i} \gamma _{i}.
\end{equation}
%
This activity is then used in various equations describing
thermodynamic equilibrium and chemical kinetics. 

The second kind of activity coefficient, the rational activity
coefficient, pertains to the solvent, water ($w$). Its activity
coefficient is denoted by $\lambda_{w}$ to emphasize that it is
different in kind: it is a mole fraction activity coefficient. The
thermodynamic activity of water ($a_{w}$) is obtained by multiplying
the mole fraction of water ($X_{w}$) by the activity coefficient of
water:
\begin{equation}
a_{w} \eq X_{w} \lambda _{w} .
\end{equation}
The activity of water is also different in kind from the activity of a
solute species (a mole fraction activity as opposed to a molal
activity).  In treating the thermodynamics of aqueous electrolyte solutions,
the activity of water and the activity of a solute species are almost
always treated as described above.

\todo[color=cyan]{GEH: Why are we including this discussion of log vs ln?}Activity coefficients are generally first calculated in logarithmic
form (e.g., $\ln \gamma_{i}$ or $\log \gamma_{i}$). In practical
usage, activity coefficients are used most often used in base-10
logarithmic form, being converted from natural logarithm form as
necessary. The conversion is illustrated by
\begin{equation}
\log\gamma_{i} =\frac{\ln\gamma_{i}}{\ln (10)}.
\end{equation}
The conversion factor $\ln(10)$ is approximately equal to 2.303, and
this value often appears in equations in the literature in place of
the exact factor. The approximate value should not be used in this
toolset, as it is insufficiently precise for accurate work. Instead
the value should be calculated using the same floating-point precision
that will be used to calculate activity coefficients. This is most
efficiently done by calculating the value once and then storing it for
subsequent use. 
It is noted that ``log'' is somewhat ambiguous, in
that the literature contains examples of it being used for both
natural and base-10 logarithms. In the present description given in this section (Section~\ref{sec:AqueousActivityCoefficients}), 
it will always refer to the base-10 logarithm.

Activity coefficient model equations ideally satisfy thermodynamic
consistency relations. The value of consistency lies in allowing the
possibility of accuracy at higher ionic strengths. Low ionic strength
models typically include inconsistent equations, but the numerical
consequences of the inconsistencies tend to be acceptable in the range
of applicability of these models. For electrolyte solutions,
\citet{wolery-1990} presents equations and methods for ensuring the
development of 
consistent equations and for testing the consistency of existing sets
of equations. The easiest means of testing for consistency is to use
the cross-differentiation rule, which takes the following forms for
solute-solute and solvent-solute pairs:
\begin{eqnarray}
\frac{\partial\ln\gamma_{j}}{\partial m_{i}} &=&\frac{\partial\ln\gamma_{i}}{\partial m_{j}}, \\
N_{w}^{kg} \, \frac{\partial \ln a_{w}}{\partial m_{i} } &=&-1-\sum _{j}m_{j} \frac{\partial \ln \gamma_{i}}{\partial m_{j}}
\end{eqnarray}
where $i$ and $j$ denote different solute
species and $N_{w}^{kg}$ is the number of moles of water in a 1 kg mass
(approximately 55.51).

\subsubsection{The Debye-H\"{u}ckel Equations} 
\label{sec:debyehuckel}

Activity coefficient model equations for electrolyte solutions
generally include some type of Debye-H\"{u}ckel term to represent the
effects of long-range electrical forces. The most common
representation is based on the ``extended'' Debye-H\"{u}ckel equation,
which for the activity coefficient of an ionic solute species is given
by
%
\begin{equation}
   \log \gamma _{i} = - A_{\gamma ,10} z_{i}^{2} 
                       \left( \frac{\sqrt{ \bar{I} }}{1 + b \sqrt{\bar{I}}} \right).
\end{equation}
%
Here $A_{\gamma ,10} $ is the Debye-H\"{u}ckel ``$A$'' parameter for the
activity coefficient (hence the subscript ``$\gamma$''), modified for
consistency with the base-10 \todo[color=cyan]{GEH: I suggest that we pick one convection for log (base-10) vs. natural log and stick with it throughtout this entire document.}logarithmic activity coefficient on the
left-hand-side of the equation (hence the additional subscript,
``10''). To assist in avoiding potential confusion, $A_{\gamma ,10}$
should have a value of 0.5114 at
25${}^\circ$C and 1.013 bar pressure. The parameter ``$b$'' is
conceptually the product of the Debye-H\"{u}ckel ``$B$'' parameter for
the activity coefficient ($B_{\gamma}$) and a length that
corresponds to either the diameter of the ion in question or a
characteristic distance of closest approach to itself or any other ion
in solution. Practical models treat this in various ways. Some assign
a constant value, typically 1.0, 1.2, or 1.5. Others use the product
of $B_{\gamma}$ (which has a known temperature and pressure
dependence) and some sort of length parameter.

The equation for the activity of water corresponding to the extended
Debye-H\"{u}ckel equation is
%
\begin{equation}  \label{eq:ActivityWater}
  \log a_{w} = \frac{1}{N_{w}^{kg} } 
                 \left( - \frac{\sum _{i}m_{i}}{\ln(10)} 
                        + \frac{2}{3} A_{\gamma ,10} \bar{I}^{3/2}
                          \varsigma(b\sqrt{\bar{I}})
                 \right).
\end{equation}
%
where the summation over molalities spans all solute
species (all aqueous species except the solvent), and the function
$\varsigma $(x) in Equation~\eqref{eq:ActivityWater} is given by
\begin{equation}
  \varsigma (x) \eq \frac{3}{x^{3} } \left(1+x-\frac{1}{1+x} -2\ln (1+x)\right),
\end{equation}
where $x$ serves the purpose of a generic variable.
If the activity coefficient of water is desired, it can be obtained
from the relation
%
\begin{equation}
   \log \lambda _{w} \eq \log a_{w} -\log (X_{w} ),
\end{equation}
%
where the mole fraction of water is given by
%
\begin{equation}
   X_{w} \eq \frac{N_{w}^{kg} }{N_{w}^{kg} +\sum _{i}m_{i}  } .
\end{equation}
%
The activity of water is closely related to the osmotic coefficient, $\varphi$:
%
\begin{equation}
   \varphi \eq -\left(\frac{N_{w}^{kg} }{\sum _{i}m_{i}  } \right)\ln a_{w} .
\end{equation}

All forms of the extended Debye-H\"{u}ckel equation are consistent with
the Debye-H\"{u}ckel Limiting Law (DHLL):
%
\begin{equation}
   \lim \limits_{\bar{I}\to 0} \log \gamma _{i} \eq -A_{\gamma ,10} z_{i}^{2} \sqrt{\bar{I}}.
\end{equation}
%
The limiting law is a critical feature describing the behavior of
ionic activity coefficients in the range of very low ionic strength.
The ionic activity coefficient plunges rapidly from unity as ionic
strength increases from zero. There is no comparable limiting relation
for the activity of water, due to the compositional dependence on both
the ionic strength and the sum of solute molalities.

In general, the extended Debye-H\"{u}ckel equation is not useful for
significant practical modeling, as it is accurate only in very dilute aqueous
solutions. If only monovalent ions are present, it may be useful for
$\bar{I} < $ 0.1 molal. In the presence of higher valence ions, the
maximum range becomes more compressed. Most practical models therefore
extend the ``extended'' Debye-H\"{u}ckel equation by adding additional
terms or otherwise adding to the mathematical complexity, in the
process introducing more model parameters.

The activity coefficient models that will be available in this toolset
include the Davies equation, the B-dot equation, Pitzer's equations,
Extended UNIQUAC, and NEA-SIT. The models are first addressed, followed by  
the discussion on rescaling the activity coefficients.

\subsubsection{The Davies Equation} 
\label{sec:davies}

The \citet{CWDavies_1962} equation is a commonly used at low ionic strength (less
than about 1 molal) model. The activity coefficient of an aqueous
solute species is given by
%
\begin{equation}
  \log \gamma _{i} \eq - A_{\gamma ,10} z_{i}^{2} 
                      \left( \frac{ \sqrt{ \bar{I} } }{1 + \sqrt{ \bar{I} } }- d \bar{I}  \right).
\end{equation}
%
Here $d$ is a constant, either 0.2 as in EQ3/6 \citep{Wolery-1992}
or 0.3 as in PHREEQC (Parkhurst and Appelo, 1999). If the
``$d \bar{I}$'' part is dropped, this equation reduces to the extended
Debye-H\"{u}ckel form with $b$ set to unity. It can be shown
that the full equation satisfies the solute-solute-form of the
cross-differentiation rule.

For the activity coefficient of water, the matching equation used in
EQ3/6 for the activity of water is
\begin{equation}
  \log a_{w} \eq \frac{1}{N_{w}^{kg} } \left(-\frac{\sum _{i}m_{i}  }{\ln (10)} 
  +
  \frac{2}{3} A_{\gamma ,10} \bar{I}^{\frac{3}{2} } \varsigma (\sqrt{\bar{I}} )-dA_{\gamma ,10} \bar{I}^{2} \right).
\end{equation}
where all parameters and the $\varsigma (x)$ function have been
previously introduced (See Section~\ref{sec:debyehuckel}). This
equation is a corrected version of that given by \citet{Wolery-1992}
(Equation 86 in that document). Here a factor of 2 in the 
``\textit{d}'' term has been
removed, and $d$ substitutes for the original constant value of
0.2. This equation is quasi-consistent with the equation for the
activity coefficient of a solution species, in the sense that the
solvent-solute form of the cross-differentiation rule is satisfied for
the case of a pure solution of a uni-univalent electrolyte, such as
sodium chloride. It does not satisfy this rule in the general case.

The equation used by PHREEQC is symbolically equivalent to
\begin{equation}
  a_{w} \eq 1 - \frac{1}{N_{w}^{kg} } \sum _{i}m_{i}.  
\end{equation}
As given by the source (Parkhurst and Appelo, 1999, p. 17), the
factor 1/$N_{w}^{kg} $ is replaced by a constant value of 0.017, which is
rather approximate, and the molality is replaced by the mole number
divided by the number of kg of solvent water (this substitution is
exact). This equation is based on ignoring the activity coefficient of
water and replacing the mole fraction with a limiting approximation of
itself. Hence, the activity of water is given in direct form, rather
than logarithmically.

For the present toolset, it is recommended that the Davies model be
implemented as two options, one (Davies-EQ3/6) consistent with the
implementation in EQ3/6, the other (Davies-PHREEQC), with
PHREEQC. This will permit direct comparison with both codes.

The Davies equation predicts a unit activity coefficient for
electrically neutral solute species. This is known to be generally
inaccurate, as the activity coefficients of non-polar neutral solutes
such as O$_2$(aq) and N$_2$(aq) should increase with ionic strength (the
``salting out'' effect), while the activity coefficients of polar
species such as CaSO$_4$(aq) and MgSO$_4$(aq) should decrease
(``salting-in'').

In practice, the Davies equation is mainly used for low temperatures
(near 25${}^\circ$C) and near-atmospheric pressures. The $A_{\gamma
,10}$ parameter has temperature and pressure dependence. As long as
this is accounted for, the Davies equation model could be applied in
principle at higher temperatures and pressures. However, it needs to
be kept in mind that the 0.2 constant was obtained by fitting data to
solutions for temperature near 25${}^\circ$C and for atmospheric
pressure. The accuracy of the model is therefore likely to deteriorate
at higher temperatures and pressures.

\subsubsection{The B-dot Equation} 
\label{sec:b_dotEquation}

The B-dot equation of \citet{helgeson-1969} is an alternative low ionic
strength model. The activity coefficient of a solute species is given
by
%
\begin{equation}
   \log \gamma _{i}  
      \eq - \frac{A_{\gamma ,10} z_{i}^{2} \sqrt{ \bar{I} }}
               {1+ \mathring{a_i} B_{\gamma } \sqrt{\bar{I}}} 
        + \dot{B} \bar{I}.
\end{equation}
%
where $\mathring{a_i}$ is the diameter of the $i^{th}$ solute species,
$B_{\gamma } $ is the Debye-H\"{u}ckel B parameter for the activity
coefficient, and $\dot{B}$ is the ``B-dot'' parameter.  Removing the
$\dot{B}\bar{I}$ term and setting $\mathring{a} B_{\gamma }$ to unity, this
equation reduces to the Davies equation with the $d\bar{I}$ term omitted.
Comparison with the Davies equation brings up two points. The first is
that the B-dot model has more parameters.  The $B_{\gamma}$ parameter
appears, and each solute species has an assigned diameter. The
``B-dot'' parameter itself is an additional parameter.

It can be shown that the B-dot equation does not satisfy the
solute-solute form of the cross-differentiation rule. There is an
issue with the first term on the right hand side in that the rule can
only be satisfied if all aqueous ions have the same diameter. There is
an issue with the second term in that the rule is only satisfied if
the charge number squared is the same for all ions, as would be the
case for example in a pure sodium chloride solution.

For an electrically neutral species, the B-dot equation reduces to
%
\begin{equation}
  \log \gamma _{i} \eq \dot{B}\bar{I}.
\end{equation}
%
As the $\dot{B}$ parameter is generally assigned a positive value, this
would provide for some measure of ``salting-out.'' By tradition,
however, the B-dot equation is not applied to neutral solute species,
and it will not be so applied in the present toolset. For non-polar
neutral species, the common practice is to assign an approximation for
the activity coefficient of CO$_2$(aq) in otherwise pure sodium chloride
solution of the same ionic strength. The approximation used in EQ3/6
(based on \citet{drummond-1981}, and which will be adopted for the present
toolset) is
%
\begin{equation}
  \ln \gamma_{i} \eq \left( C + FT + \frac{G}{T} \right) I 
                  -( E + HT ) \left( \frac{\bar{I}}{\bar{I} + 1} \right).
\end{equation}
%
Here T is the absolute temperature and C = -1.0312, F = 0.0012806, G =
255.9, E = 0.4445, and H = -0.001606. Note that the result is
presented in terms of the natural logarithm.
For a polar aqueous species, the EQ3/6 practice (which will be adopted
in the present toolset) is to use
%
\begin{equation}
\log \gamma _{i} \eq 0.
\end{equation}
%
Because different equations are used for electrically neutral solute
species than for ionic species, there is necessarily an additional set
of violations of the solute-solute cross-differentiation rule.

For the activity of water, the B-dot model as implemented in EQ3/6
(and recommended for the present toolset) is to use the equation
%
\begin{equation}
   \log a_{w} 
   = \frac{1}{N_{w}^{kg}} 
     \left( - \frac{\sum _{i}m_{i}}{\ln (10)} 
            + \frac{2}{3} A_{\gamma ,10} \bar{I}^{\frac{3}{2}} 
              \sigma( {\mathring{a}} B_{\gamma} \sqrt{\bar{I}} )
            - \dot{B}\bar{I}^{2} 
     \right).
\end{equation}
%
All the parameters here have been introduced previously, except for
the unsubscripted $\mathring{a}$, which is conceptually an effective solute species
diameter. In practice, this is assigned a constant value of 4.0
angstroms.

The above equation for the activity of water is quasi-consistent with
the solvent-solute form of the cross-differentiation rule. The term
containing the effective solute diameter leads to inconsistency unless
every ionic solute has a matching diameter value. The term containing
$\dot{B}$ leads to inconsistency unless the solution is a pure
solution of a uni-univalent electrolyte such as sodium chloride.

The thermodynamic inconsistencies noted above introduce some level of
inaccuracy into the model, tending to negate the improvement that
might be expected by introducing more parameters (e.g., a diameter
value specific to each solute species). Thus, for temperature near
25${}^\circ$C and near-atmospheric pressure, the B-dot model is
probably as good as the Davies equation model.

The B-dot model does have an advantage over the Davies equation model
in that it is better parameterized to cover a wide range of
temperature and pressure. In addition to $A_{\gamma ,10}$, the
$B_{\gamma} $and $\dot{B}$ parameters are treated as functions of
temperature and pressure. The $A_{\gamma ,10} $ and $B_{\gamma}$ 
parameters have values derived from pure theory (and models for pure
water properties). The $\dot{B}$ parameter is obtained by fitting to
data for pure sodium chloride solutions. The ion size parameters are
treated as constant with respect to temperature and pressure.

In regard to solute species, diameters are only necessary for ionic
species. Some means needs to be provided to specify (as on a
supporting thermodynamic data file) whether a neutral species is to be
treated as non-polar or polar. All the necessary information could be
folded into a diameter array or equivalent structure, in which the
values in the case of neutral species would not be actual diameters,
but code values specifying non-polar or polar type. However, a
separate flagging structure should be utilized, as the variable type
can then be something more appropriate (integer or logical) than the
floating point necessary for actual diameters. Also, the structure for
diameters would then be free to include diameters for neutral solute
species. Although such diameters are not be used in the B-dot model,
they might be usable in other models.

\include{Pitzer}

\include{UNIQUAC}

\include{NEA-SIT}

\subsubsection{Rescaling Ionic Activity Coefficients}

The activity coefficient models described above include descriptions
of individual ion activity coefficients. This is problematic in that
ionic activity coefficients and ionic activities are not measurable
for individual ions. These quantities are measurable only in
combinations that correspond to electrical neutrality. For activity
coefficients, examples of such combinations include 
$\log \gamma_{H^{+} } +\log \gamma_{Cl^{-} }$ and 
$2\log \gamma_{H^{+} } +\log \gamma_{SO_{4}^{2-}}$; 
examples for activities are analogous.
Molalities of individual ions are measureable (or quantifiable by
inference). Thus, if one could obtain or specify the activity or
activity coefficient of one single ion in an aqueous solution, one
could then use this as a reference to obtain the activities and
activity coefficients of all other ions present in the same solution.

The need to define activity coefficients and activities for individual
ionic species is dealt with by the use of a ``splitting'' convention.
Such a convention is at least somewhat arbitrary, although it may be
guided in part by theoretical concerns. One could address the issue by
adopting the results of model equations for single ion activity
coefficients. The model equations for these are all in part arbitrary,
implicitly including a splitting based on some combination of
theoretical notions and pleasing (but not necessarily unique)
symmetry. The problem with just using the model equations in their
native form is that other conventions have been previously adopted
into measurement practice, particularly the measurement of pH. For
accurate modeling consistent with standard analytical chemistry
practice, it becomes necessary to rescale the results of the model
equations presented above. This only affects the activity coefficients
of ionic species. For most analytical splitting conventions, some
expression is specified for the activity coefficient of a reference
ion, usually Cl$^-$ or H$^+$.

The most significant analytical splitting conventions for aqueous ions
are tied to the definition of the pH. Conceptually,
%
\begin{equation}
   pH = -\log a_{H^{+}} .
\end{equation}
%
In order to provide a practical basis for measuring the pH, it is
necessary to define the activity of the hydrogen ion. The splitting
convention used for this purpose then defines a pH scale. The choice
of pH scale further affects the definition of the redox potential, E$_h$. %\todo{subscript h? - Williamson}

In modern work, the dominant pH scale is the NBS scale, originated by
the National Bureau of Standards, now the National Institute of
Standards and Technology. The NBS scale is based on the
Bates-Guggenhiem equation \citep{bates-1964}:
%
\begin{equation}
  \log \gamma _{Cl^{-} } =-\frac{A_{\gamma ,10} \sqrt{\bar{I}}}{1+1.5\sqrt{\bar{I} } } .
\end{equation}
%
This is a simple form of the extended Debye-H\"{u}ckel equation. It
defines the activity coefficient of the chloride ion. It may be
surprising that chloride is used as the reference ion rather than the
hydrogen ion, which is more closely tied to the pH.  What is apparent is that the Bates-Guggenheim
equation must give a result that is different from what would be
obtained for the chloride ion using say the Davies equation or the
B-dot equation, or for that matter, from Pitzer's equations. In the
range of low ionic strength (say less than 1 molal), the differences
should be numerically small for each of the three practical models, as
they and the Bates-Guggenheim equation all include some form of
extended Debye-H\"{u}ckel model and thus are all consistent with the
Debye-H\"{u}ckel limiting law.  At higher ionic strength, however, the differences can be substantial (the equivalent of several pH units). 

The Bates-Guggenheim equation can be applied whether or not there is
any chloride in aqueous solution, as the equation is sufficient to
calculate the specified activity coefficient. The charge number of -1
is effectively built into the equation. 

The Bates-Guggenheim equation (the NBS pH scale) is effectively built
in to the calibration of all modern means of measuring the pH, whether
in pH calibration buffers or pH paper. EQ3/6 for example by default
rescales ionic activity coefficients computed from the models to be
consistent with the NBS pH scale (other options, including no
rescaling, may be offered). Rescaling from one scale (scale ``1'') to
another (scale ``2'') is accomplished using 
%
\begin{equation}
   \log \gamma_{i}^{(2)} 
     = \log \gamma_{i}^{(1)} 
     + \frac{z_{i}}{z_{j}} 
       \left( 
         \log \gamma_{j}^{(2)} - \log \gamma_{j}^{(1)} 
       \right).
\end{equation}
%
Here $j$ denotes the reference ion (here Cl$^-$) and $i$ denotes any ion
(including the reference ion). In the present context, scale ``1'' is
usually that implied by a model equation and scale ``2'' is the
desired scale.

An alternative convention is to choose
%
\begin{equation}
  \log \gamma _{H^{+}} = 0.
\end{equation}
%
For the hydrogen ion, this results in
% 
\begin{equation}
  pH = -\log m_{H^{+}} .
\end{equation}
%
as the activity and molality of the hydrogen ion are numerically
equivalent. The rescaling of ionic activity coefficients for
consistency does not give an analogous result for other ions. EQ3/6
allows rescaling using this convention as an option, but it has
limited utility and it not recommended as a general option in the
present toolset.

The definition of the pH in terms of molality (``pmH'') is significant independent of rescaling. Thus one has simply

\begin{equation}
  pmH = -\log m_{H^{+} } .
\end{equation}
%
In concentrated electrolyte solutions (e.g., WIPP, Hanford tanks), pmH
is often more useful for assessing the acidity/basicity of a solution
than the NBS pH. The NBS pH cannot be accurately measured in
concentrated solutions owing to liquid junction effects with
electrodes and interferences with dyes. Also, the Bates-Guggenheim
equation (and the NBS pH scale itself) was originally intended for use
only at low ionic strength. \citet{bates-1964} suggested application
to solutions with ionic strengths of no greater than 0.1 molal. Since
then, however, the scale has been used at higher ionic strengths. This
has led to the problem that of two highly concentrated solutions with
an NBS pH of say 2, one might be acidic (in the sense that H+ is
abundant) and the other not (in the sense that H+ is not abundant). In
other words, the common association of pH values with various degrees
of acidity/basicity (e.g., 7 is neutral) no longer applies.

Still other conventions and scales exist. However, for the present
toolset only the following is required. First, the default behavior
will be to apply rescaling to the NBS scale. Second, the option will
be available to use the basic model results without rescaling. Third,
the pmH will be directly calculated and included in the output. An
option to rescale the activity coefficients for consistency with the
$\log \gamma _{H^{+}} = 0$ convention will not be required.

 
